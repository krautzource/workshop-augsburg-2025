\documentclass[a4paper]{amsart}

% macro
\newcommand{\R}{\mathbb{R}}

\usepackage{amsmath}
\usepackage{hyperref}

\begin{document}
Aus der \textit{amsmath} Dokumentation.

\begin{equation}\label{xx}
    \begin{split}
    a& =b+c-d\\
    & \quad +e-f\\
    & =g+h\\
    & =i
    \end{split}
\end{equation}

\begin{equation}
    \begin{split}
        a& =b+c-d\\
         & \quad +e-f\\
         & =g+h\\
         & =i
        \end{split}
\end{equation}

\begin{equation*}
    \left.\begin{aligned}
    B'&=-\partial\times E,\\
    E'&=\partial\times B - 4\pi j,
    \end{aligned}
    \right\}
    \qquad \text{Maxwell's equations}
\end{equation*}


\begin{flalign}
    x&=y & X&=Y\\
    x'&=y' & X''&=Y'\\
    x+x'&=y+y' & X+X'&=Y+Y'
\end{flalign}


\begin{equation}
    f(x) = x^3 + 2x^2 + x \tag{*}\label{theFunction}
\end{equation}

In Formel \ref{theFunction} sehen wir...

Ein Macro $\R$.

\end{document}
