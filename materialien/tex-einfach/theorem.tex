\documentclass[a4paper]{amsart}

\usepackage{amsmath}
\newtheorem{theorem}{Satz}[section]
\newtheorem{corollary}{Korollar}[theorem]
\newtheorem{lemma}[theorem]{Lemma}
\usepackage{hyperref}


\begin{document}

\section{Theorem Umgebungen}
\begin{theorem}
    Sei \(f\) eine Funktion, \dots.
\end{theorem}

\begin{theorem}[Pythagorean]
\label{pythagorean}
Vereinfacht:
\[ x^2 + y^2 = z^2 \]
\end{theorem}
    
Aus \ref{pythagorean} folgt    

\begin{corollary}
    Kein rechtwinkliges Dreieck hat Seitenlängen 3cm, 4cm, und 6cm.
\end{corollary}
    

\begin{lemma}
Lemmata sind oft bedeutsamer als Sätze.
\end{lemma}
\end{document}
